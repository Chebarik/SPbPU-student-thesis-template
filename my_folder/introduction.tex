\chapter*{Введение} % * не проставляет номер
\addcontentsline{toc}{chapter}{Введение} % вносим в содержание

В современных практиках непрерывной доставки (CI/CD) построение гибких, масштабируемых и надёжных систем управления процессами развертывания является критически важной задачей для организаций, стремящихся обеспечить высокую скорость выпуска программного обеспечения при сохранении стабильности и безопасности \cite{microsoft-cd, devops-metrics}. Предметом исследования является архитектура бэкенд-системы управления процессами развертывания, построенная на основе формального аппарата конечных автоматов, с интегрированными подсистемами управления пакетами артефактов и выполнения скриптов на рабочих машинах. Основное внимание нацелено на архитектурные решения и теоретические основы, позволяющие описывать сложные процессы в декларативном виде, обеспечивать строгий контроль переходов между состояниями (включая механизмы ручных одобрений и блокировок) и поддерживать полный жизненный цикл артефактов от сборки до развертывания \cite{fsm-habr, fsm-allegro, bpmn-spec}.

Актуальность исследования обоснована возрастающей сложностью современных процессов развертывания, когда число этапов конвейера, количество интеграций с внешними системами и требования к прослеживаемости и аудиту непрерывно растут. Традиционные CI/CD-платформы, такие как Jenkins, GitHub Actions и GitLab CI/CD, опираются преимущественно на императивное или декларативное описание последовательности шагов, однако зачастую не предоставляют формальной модели для описания сложных ветвлений, циклов и условных переходов между состояниями процесса \cite{jenkins-handbook, github-actions-docs, gitlab-cicd-docs}. Отсутствие строгой формальной основы затрудняет верификацию корректности процессов, приводит к сложностям в тестировании и сопровождении конвейеров, а также увеличивает риск ошибок при изменении логики развертывания. Кроме того, при масштабировании числа процессов и артефактов возникают проблемы с производительностью репозиториев пакетов, эффективностью индексации метаданных и согласованностью распределённого кэширования \cite{octopus-packages, github-packages}.

Значимость исследования определяется тем, что применение теории конечных автоматов для моделирования процессов развертывания позволяет формализовать спецификацию workflow, обеспечить корректность переходов и упростить визуализацию и редактирование сложных процессов. Конечный автомат (КА) представляет собой математическую модель дискретной системы с конечным числом состояний и переходами между ними \cite{hopcroft-ullman, fsm-habr}. Графовое представление состояний и переходов обеспечивает наглядность, упрощает анализ достижимости состояний и позволяет применять формальные методы проверки свойств (например, отсутствие тупиковых состояний, завершимость процесса). Практическая ценность применения формального аппарата состоит в возможности поддержки как простых линейных конвейеров, так и сложных сценариев с параллельным выполнением задач, условными ветвлениями и механизмами ручного вмешательства на критических этапах \cite{aws-codepipeline-approval, azure-devops-approvals}.

Анализ существующих решений показывает разнообразие подходов к организации систем управления процессами развертывания. Octopus Deploy представляет собой платформу с развитой подсистемой управления релизами, пакетами и процессами, демонстрирующую практическое применение концепций управлений состояниями и оркестрации workflow \cite{octopus-packages}. AWS CodePipeline предлагает модель с поддержкой ручных утверждений, но ограничен интеграцией только с экосистемой AWS \cite{aws-codepipeline-approval}. GitHub Actions предоставляет гибкость через декларативное YAML-описание workflow, однако использует преимущественно императивный подход к описанию последовательности шагов \cite{github-actions-docs}. Систематизация архитектурных решений таких систем представляет практический интерес для выявления общих паттернов проектирования и формирования обоснованных рекомендаций.

\textbf{Целью} исследования является систематизация теоретических основ и практических подходов к построению архитектуры систем управления процессами развертывания на основе конечных автоматов и формирование требований к компонентам таких систем.

Для достижения цели решаются следующие задачи:
\begin{enumerate}
    \item Исследование и систематизация теоретических основ описания процессов на базе конечных автоматов, формальных свойств workflow и подходов к описанию состояний и переходов.
    \item Сравнительный анализ существующих платформ для управления процессами развертывания и оркестрации с выявлением архитектурных решений и ограничений.
    \item Исследование практических моделей и форматов представления процессов (YAML, JSON, DSL) и их сопоставление с теоретическими моделями конечных автоматов.
    \item Анализ подходов к управлению пакетами артефактов, включая форматы упаковки, системы хранения, версионирование и стратегии распределённого кэширования метаданных.
    \item Исследование методов выполнения скриптов на исполняющих машинах и анализ технологий для потоковой передачи логов и мониторинга.
    \item Формирование требований к API взаимодействия подсистем и архитектуре компонентов системы управления процессами и пакетами на основе результатов анализа.
\end{enumerate}

%% Вспомогательные команды - Additional commands
%\newpage % принудительное начало с новой страницы, использовать только в конце раздела
%\clearpage % осуществляется пакетом <<placeins>> в пределах секций
%\newpage\leavevmode\thispagestyle{empty}\newpage % 100 % начало новой строки